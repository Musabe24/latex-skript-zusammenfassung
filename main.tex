\documentclass[10pt]{article}

% --- Pakete ---
\usepackage[utf8]{inputenc} % Für deutsche Umlaute
\usepackage[ngerman]{babel} % Deutsche Anpassungen
\usepackage{amsmath, amssymb} % Mathematische Symbole und Umgebungen
\usepackage{graphicx}        % Für Bilder
\usepackage{geometry}        % Seitenränder anpassen
\usepackage{enumitem}       % Für Listen
\usepackage{hyperref}       % Für anklickbare Verweise (Inhaltsverzeichnis)
\usepackage{xcolor}         % Für Farben

% --- Seitenränder ---
\geometry{a4paper, left=2.5cm, right=2.5cm, top=2.5cm, bottom=2.5cm}

% --- Titelseite ---
\title{Vorlesungsskript: [Name der Vorlesung]}
\author{Dein Name}
\date{\today}

% --- Eigene Befehle  ---
\newcommand{\wichtig}[1]{\textbf{\textcolor{red}{#1}}} % Wichtige Begriffe hervorheben

% --- Dokumentenbeginn ---
\begin{document}

\maketitle % Titelseite anzeigen
\tableofcontents % Inhaltsverzeichnis

\newpage % Neue Seite nach dem Inhaltsverzeichnis

% --- Kapitel 1 ---
\section{Thema des Kapitels}

\subsection{Thema des Unterpunkts}

Zusammenfassung des ersten Unterpunkts.

\begin{itemize}
    \item Wichtiger Punkt 1
    \item Wichtiger Punkt 2
\end{itemize}

\textbf{Definition:} Hier Definition einfügen

\textbf{Formel:}
\[
    E = mc^2
\]

\subsection{Thema des Unterpunkts}

...

% --- Kapitel 2 ---
\section{Thema des Kapitels}

...

% --- Anhang ---
\appendix
\section{Formelsammlung}

Hier können wichtige Formeln übersichtlich aufgelistet werden.

\section{Glossar}

Hier können wichtige Begriffe erklärt werden.

\section{Literatur}

Hier kann relevante Literatur zum Nachlesen erwähnt werden.

% --- Dokumentenende ---
\end{document}
